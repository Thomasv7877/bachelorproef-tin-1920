%---------- Inleiding ---------------------------------------------------------

\section{Introductie} % The \section*{} command stops section numbering
\label{sec:introductie}

Het IT Team van AZ Glorieux heeft oog op de toekomst. Twee grote voorbeelden hiervan zijn een push om nodige toestellen te migreren naar Windows 10 en een geleidelijke adoptie van Intune als aanvulling op System Center Configuration Manager (SCCM). Tijdens mijn stage heb ik hier een bescheiden bijdrage aan kunnen leveren. Het beoogde onderwerp voor mijn bachelorproef is verbonden met deze vernieuwing. LanReview, een binnenshuis ontwikkeld assetmanagement programma dat onontbeerlijk is voor de IT Helpdesk, is aan vernieuwing toe. De laatste ontwikkeling is van enige tijd geleden en het is niet logisch meer om dit terug op te nemen voornamelijk wegens de verouderde code base (Visual BASIC 3.0). De vervangende applicatie moet gemaakt worden met Microsoft Power Apps en ondersteunend ook Power Automate.

Meer achtergrond over de IT voorziening van AZ Glorieux verklaart deze keuze. Het is een groot regionaal ziekenhuis met een netwerk dat meer dan 1000 toestellen telt. Het domein is gebouwd met System Center Configuration Manager. Zowel Clients als Servers gebruiken vormen van het Microsoft Windows besturingsysteem. Microsoft Office wordt op de meeste pc's gebruikt. Power Apps is ook opgenomen in het Office 365 plan.

\vspace{5mm}

Een overzicht van de functies van LanReview die zeker ook in zijn opvolger aanwezig dienen te zijn:
\begin{itemize}
    \item  Overzicht geven van belangrijke informatie voor elke toestel in het domein.
    \item Rapporten kunnen genereren.
\item Pc’s bedienen vanop afstand.
\end{itemize}
Er zijn ook enkele volledig nieuwe functies beoogd:
\begin{itemize}
\item Mobiel bruikbaar zijn.
\end{itemize}
Overige functionaliteit:
\begin{itemize}
    \item Grotere mate van Automatisatie introduceren met Power Automate.
    \item Buiten het domein bruikbaar zijn.
    \item Nieuwe types toestellen opnemen.
    \item Relevante Randapparatuur opnemen, deze in relatie stellen met de gekoppelde PC.
    \item Revisie van datavelden per toestel.
    \item Diverse Grafische User Interface (GUI) verbeteringen: kleuren gebruiken, tabbladen.
    \item Met barcodes kunnen werken. Bijvoorbeeld de mac adres barcode inscannen om een nieuwe entry in de applicatie te maken.
\end{itemize}
Er zal een requirementsanalyse uitgevoerd worden via de MoSCoW-methode \parencite{Wikipedia2020}. Dit staat verder beschreven in \nameref{sec:methodologie}.

\vspace{5mm}

Verdere uitleg van de hoofdfunctionaliteiten van LanReview via enkele use cases:
\begin{itemize}
    \item Er komt een telefoon binnen van een dokter die een probleem heeft met één van de medische programma's. Het computernummer wordt doorgegeven. We zoeken op dit nummer in LanReview en krijgen een overzicht van de PC in kwestie. Het is nu mogelijk te rechtsklikken en de pc over te nemen om het probleem verder op te lossen.
    \item Er werden nieuwe laptops aangekocht. We nemen deze op in LanReview door op zijn minst een niet-toegewezen computernummer te kiezen samen met het MAC adres. Aanvullende zaken zijn: een beschrijving, model, locatie, IP-nummer.
    \item Een model laptop dient geüpdatet te worden naar Windows 10. Binnen LanReview queryen we op dit model en krijgen we een overzicht van elk toestel terug.
\end{itemize}

\vspace{5mm}

De hoofddoelstelling bestaat uit het opstellen van een competente Proof of Concept. Hiernaast moet verdere ontwikkeling zo toegankelijk mogelijk gemaakt worden. Met andere woorden de proef en bijbehorende documenten moeten een hoge mate van praktisch nut hebben voor het IT Team om niet-gerealiseerde functionaliteit in de proof of concept en toekomstige functionaliteiten toe te kunnen voegen.\\
De onderzoeksvragen liggen in lijn met het bovenstaande:
\begin{itemize}
    \item Is het mogelijk een vervanger voor LanReview te bouwen met Power Apps die elk de vier hoofdfunctionaliteiten ondersteunt en op zijn minst drie vierde van de overige functionaliteiten kan ondersteunen?
    \item Is Power Apps werkelijk de beste keuze hiervoor of is er meerwaarde in een volledig gerealiseerd IT asset management pakket? Is er alternatief een beter geschikt low-code platform?
    \item Is de Proof of Concept eenvoudig uit te breiden met nieuwe functionaliteiten? Is dit aanvaardbaar voor het IT team van AZ Glorieux?
    \item Is Power Apps robuust genoeg om meer complexe functionaliteit te ondersteunen? Is het met een zelfgeschreven uitbreiding bijvoorbeeld mogelijk om de remote desktop functionaliteit te verwezenlijken?
    \item Is de gebruikte methode op zijn minst deels bruikbaar om andere applicaties voor de IT van AZ Glorieux te bouwen? Er worden twee cases onderzocht: één voor een telefoonboek (legacy business applicatie met een lage moeilijkheidsgraad) en één om het potentieel van Power Apps te demonstreren. Dit is opgenomen in de volgende onderzoeksvraag.
    \item Is er een use case voor nieuwe of experimentele functionaliteiten in Power Apps zoals integratie met Teams of beperkt toepassen van AI via de AI Builder?
    \item De Proof of Concept zal nauw samenwerken met SCCM, is het concreet mogelijk om in de nabije toekomst ook samen te werken met Intune?
\end{itemize}


%---------- Stand van zaken ---------------------------------------------------

\section{State-of-the-art}
\label{sec:state-of-the-art}

% Voor literatuurverwijzingen zijn er twee belangrijke commando's:
% \autocite{KEY} => (Auteur, jaartal) Gebruik dit als de naam van de auteur
%   geen onderdeel is van de zin.
% \textcite{KEY} => Auteur (jaartal)  Gebruik dit als de auteursnaam wel een
%   functie heeft in de zin (bv. ``Uit onderzoek door Doll & Hill (1954) bleek
%   ...'')

LanReview is eigenlijk een view op SCCM. Het is uit die databank dat de gegevens per toestel komen. Deze gegevens werden aangevuld met extra datavelden en opgeslagen in SharePoint lijsten. Om te rapporteren worden SQL-achtige query's gebruikt op deze lijsten. Het is mogelijk om op elk veld van de entries te filteren. De data van elke entry kan ook aangepast worden. Er is een optie aanwezig om pc's over te nemen. Het onderliggende remote desktopprotocol wordt opgeroepen vanuit LanReview om dit te verwezelijken.

De term assetmanagement werd gebruikt om LanReview te beschrijven maar waar LanReview een view is op SCCM kan een traditioneel IT Asset Management Program meestal gezien worden als alternatief voor SCCM. In dergelijke programma's is buiten inventarisatie ook voorziening voor netwerk discovery, analytics, security en meer. Gartner \parencite{Gartner2020} beschrijft het als volgt: \emph{IT asset management (ITAM) provides an accurate account of technology asset lifecycle costs and risks to maximize the business value of technology strategy, architecture, funding, contractual and sourcing decisions.}

Het IT asset management landschap is al enige tijd in evolutie door verspreiding van smartphone technologie en  IoT \parencite{Badnakhe2020}, dit maakt de keuze voor een nieuwe en flexibele technologie als Power Apps logisch.

PowerApps is een low-code ontwikkelingsplatform dat niet-programmeurs toelaat om business apps te maken. Het is mogelijk om canvas of modelgestuurde applicaties te bouwen \parencite{Knight2019}. Indien iets niet visueel gedaan kan worden is gebruik van een Excel-achtige querytaal mogelijk \parencite{Owen2019}. Uitbreiding is mogelijk via Connectoren naar externe services. Uitbreidingen zijn ook volledig zelf te bouwen in C Sharp \parencite{Vivek2019}.
Recente innovaties binnen het Power platform zijn de introductie van AI mogelijkheden aan de hand van Virtual Agents en een grotere integratie met Teams \parencite{Cunningham2019}. Vooral dit laatste is interessant omdat AZ Glorieux recent ook Teams is beginnen introduceren.

Automatisatie in een PowerApp kan verwezenlijkt worden via Power Automate, dat recent een naamswijziging heeft ondergaan van Flow \parencite{Weare2019}. Volledig binnen de GUI is het mogelijk een automatisatieproces te maken bestaand uit aaneengeschakelde acties en condities.

Een populaire databron voor een PowerApp is SharePoint. Integratie gebeurt eenvoudigweg via de Connector ervoor. \parencite{Owen2019a}


%---------- Methodologie ------------------------------------------------------
\section{Methodologie}
\label{sec:methodologie}

Er gaan twee onderzoekstechnieken gebruikt worden: een Proof of Concept voor de vervanger van LanReview gebouwd met PowerApps en een vergelijkende studie tussen beide applicaties. De vergelijking zal focussen op analyseren van requirements, bekijken hoe deze uitgewerkt kunnen worden en hierna ook illustrerend vergelijken met behulp van simulaties.

\vspace{5mm}

Er zal gewerkt worden in fasen. 
\begin{enumerate}
    \item De Basics: Een globaal beeld schetsen van de IT omgeving van AZ Glorieux en hierna uitdiepen wat LanReview nodig heeft om te kunnen werken (specifiek de gebruikte databronnen). Achtergrondinformatie toereiken voor de gevonden technologieën en de technologieën die we zullen gebruiken om onze POC te bouwen (Power Apps, Power Automate).
    \item LanReview Reviewen: Hiermee bedoeld het interne werken van LanReview in kaart brengen tot in de nodige details. Helaas zullen we hierbij niet op codeniveau geraken omdat de broncode niet beschikbaar is. Dit betekent ook dat geen code overgenomen kan worden naar de POC maar dat enige originaliteit nodig zal zijn bij het uitwerken van de functionaliteiten. Deze en de vorige stap zal ondersteund worden met afbeeldingen. Er zal in de eerste stappen gefocust worden op toegankelijkheid en duidelijkheid.
    \item Requirementsanalyse: Met behulp van de MoSCoW-methode gaan er prioriteiten gesteld worden in het aanzienlijk aantal beoogde functionaliteiten. De basis hiervan werd reeds gelegd tijdens gesprekken specifiek hiervoor gehouden tijdens de stageperiode met de co-promotor en collega's die de applicatie praktisch zouden gaan gebruiken.
    \item Alternatieven: Hoewel er hoogstwaarschijnlijk niet afgeweken zal worden van PowerApps zal de nodige aandacht gegeven worden aan alternatieven. Welke volledig uitgewerkte proprietary softwarepakketten bestaan er die LanReview kunnen vervangen? Uit de vergelijking van Finances Online blijkt dat er keuze genoeg is \parencite{FinancesOnline2020}. Is er serieuze concurrentie voor PowerApps vanuit het Open Source kamp? Binnen PowerApps: is SharePoint de juiste technologie om onze data op te slaan? De bedoeling hiervan is hoofdzakelijk inspiratie op te doen voor het uitwerken van de Proof of Concept. Als er echter uit dit onderzoek een superieure oplossing gevonden wordt zal dit voorgelegd worden aan de opdrachtgever.
   \item Bouwen van de Proof of Concept: voor de praktische ontwikkeling van de POC laten we ons leiden door de voorganger en door de gebruikelijke technieken opgedaan in stap 1. De testfase zal uitvoerig zijn, de POC moet namelijk op termijn inzetbaar zijn binnen AZ Glorieux. Testen van code is niet aan de orde in een low-code ontwikkelplatform buiten de zelfgeschreven uitbreidingen ervoor. Geautomatiseerd de UI testen is ook nog niet mogelijk doordat het Power Apps test Framework nog in de experimentele fase zit \parencite{Microsoft2020a}. Tests zullen dus manueel opgesteld en uitgevoerd moeten worden. Indien toegelaten zou het ook behulpzaam zijn als de eindgebruikers bij tussenmomenten de Proof of Concept uitprobeerden en feedback gaven over hun ervaring.
    \item Een vergelijking tussen LanReview met de POC. Wat zijn de gelijkenissen en verschillen? De uitwerking van belangrijke functionaliteiten zal onderling vergeleken worden. In het bijzonder: "Zijn er beperkingen in Power Apps gevonden waardoor ingeboet werd aan functionaliteit?"
    \item Aandacht voor de toekomst. De uitbreidingsmogelijkheden moeten gepeild worden. Extra functionaliteit moet zo vlot mogelijk geïntroduceerd kunnen worden door het IT Team. Is het mogelijk PowerApps elders toe te passen in AZ Glorieux via een soortgelijke methode als voor onze POC. Mogelijk wordt dit uitgelegd via een specifieke case. Hoe moet de POC aangepast worden om op termijn samen te kunnen werken met Intune zoals het nu nauw met SCCM samenwerkt? In dit gedeelte of één apart ervoor zal ook een praktisch aspect aanwezig moeten zijn voor het IT Team van AZ Glorieux dat als documentatie zal moeten dienen of alternatief, verwijzingen moet hebben naar bestaande documentatie.
\end{enumerate}


\vspace{5mm}

Praktischere zaken:\\
Wat tools betreft moet specifiek een PowerApps ontwikkelomgeving opgezet worden. Dit kan via de webapplicatie voor PowerApps als deel van het Office 365-pakket. Alternatief zou het ook mogelijk moeten zijn om het softwarepakket "PowerApps Studio" te installeren. Als het maken van klassediagrams niet uitgebreid mogelijk zou zijn vanuit PowerApps zal Visual Paradigm gebruikt worden. 

Over AZ Glorieux, een belangrijke noot:\\
Als tijdens het uitwerken van de proef het AZ Glorieux netwerk of iets ermee te maken nodig zou zijn werd reeds voorgesteld dat ik ter plaatse mag werken.

%---------- Verwachte resultaten ----------------------------------------------
\section{Verwachte resultaten}
\label{sec:verwachte_resultaten}

Tastbare resultaten:
\begin{itemize}
    \item Een comprehensieve requirementsanalyse. Als hier niet de nodige aandacht wordt gegeven bestaat het risico van de "verkeerde" POC te bouwen. Een goede basis is belangrijk dus er wordt verwacht dat de requirementsanalyse uitgebreid is.
    \item Logischerwijze zullen niet alle requirements opgenomen kunnen worden in de POC, toch wordt verwacht dat de belangrijkste uitgewerkt zijn. Ruimer gezien wordt verwacht dat de POC met weinig aanpassing de originele applicatie, LanReview, kan gaan vervangen.
    \item Een bijkomend resultaat is een set instructies of een stappenplan, de vorm kan nog wijzigen maar er zal een document gebouwd worden dat het IT Team kan gebruiken om verder te werken met PowerApps, om ontwikkeling van de POC over te kunnen nemen.
\end{itemize}


\vspace{5mm}

Met betrekking tot de POC:\\
Er wordt verwacht dat het opzetten van een PowerApps applicatie weinig problemen geeft, dit is uiteindelijk het doel van low-code platformen zoals PowerApps. Er wordt wel enige uitdaging verwacht bij het uitwerken van functionaliteit die niet direct voorzien is. Uitbreidingen schrijven, vooral als het op automatisatie aankomt, kan zich moeilijk tonen. Er moet rekening mee gehouden worden in de planning.\\
Er wordt overigens verwacht dat er niet afgeweken zal worden van een combinatie van PowerApps, Power Automate en SharePoint voor het bouwen van de POC.\\
Indien dit gemeten wordt zal er vermoedelijk geen significant verschil zijn in performantie tussen de POC en LanReview.

%---------- Verwachte conclusies ----------------------------------------------
\section{Verwachte conclusies}
\label{sec:verwachte_conclusies}

De conclusies, geformuleerd als antwoorden op de onderzoeksvragen, zijn als volgt:
\begin{itemize}
    \item Er wordt verwacht dat een vervanger voor LanReview gebouwd kan worden met LanReview. SharePoint kan gebruikt blijven als databron maar SCCM (dus SQL Server) verbinding is mogelijk beperkt door licentiewijzigingen. \parencite{Microsoft2020}
    \item Power Apps is de beste keuze in ons scenario. Zelfs als hiervoor betere software gevonden wordt zal de bijkomende kostprijs niet verantwoord kunnen worden.
    \item Het is mogelijk de POC af te leveren op een manier waarmee verdere ontwikkeling ondersteund wordt.
    \item PowerApps is robuust genoeg om de hoofdfunctionaliteiten te ondersteunen, al dan niet met zelfgemaakte uitbreidingen.
    \item Er wordt verwacht dat de gebruikte techniek ook andere toepassingen kan vinden binnen AZ Glorieux.
    \item Een aanpassing naar Intune in plaats van of in combinatie met SCCM is doenbaar. De nodige tijd en moeite investering is aanvaardbaar.
\end{itemize}

\vspace{5mm}

Samenvattend wordt verwacht dat de POC een meerwaarde is voor AZ Glorieux en dat deze kan groeien tot een waardige opvolger voor LanReview. 

