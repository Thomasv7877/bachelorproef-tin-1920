%%=============================================================================
%% Inleiding
%%=============================================================================

\chapter{\IfLanguageName{dutch}{Inleiding}{Introduction}}
\label{ch:inleiding}

Het IT Team van AZ Glorieux heeft een oog op de toekomst. Grote voorbeelden hiervan zijn een push om nodige toestellen te migreren naar Windows 10, een geleidelijke adoptie van Intune en de vervanging van een handvol oude business applicaties. Dit onderzoek staat in het teken van deze vernieuwingsslag, specifiek zal er getracht worden een bijdrage te leveren in de context het laatste voorbeeld.\\
LanReview, een binnenhuis ontwikkeld assetmanagement programma dat onontbeerlijk is voor de IT Helpdesk, is aan vernieuwing toe. De laatste ontwikkeling is van enige tijd geleden en het is niet logisch meer om dit terug op te nemen voornamelijk wegens de verouderde codebase (Visual BASIC 3.0). De vervangende applicatie moet bij voorkeur gemaakt worden met Microsoft Power Apps en ondersteunend ook Power Automate. De vraag wordt met andere woorden gesteld wat voor potentieel dit platform heeft en hoe het past in hun omgeving.\\
Dit is niets speciaals. Bedrijven kampen al jaren met een niet in te vullen vraag naar software in dit zal in de toekomst alleen maar blijven toenemen. Low-code platformen (waaronder het Microsoft Power platform) stellen zichzelf als het beste gereedschap om dit probleem aan te pakken.\\
Dat is wat onderzocht zal worden. Specifiek wordt gekeken hoe gepast het PowerApps gereedschap is om het LanReview probleem op te lossen. Dit wordt gedaan aan de hand van een proof-of-concept.

Het is belangrijk om te weten dat het onderzoek zicht beperkt tot low-code. Er zal bijvoorbeeld geen proof-of-concept op de traditionele manier geprogrammeerd worden.

Kennis van onderstaande technologieën is een meerwaarde maar niet verplicht om de tekst te kunnen begrijpen:
\begin{itemize}
    \item SQL
    \item REST API's
    \item C\# en het ASP.NET Framework.
\end{itemize}

\section{\IfLanguageName{dutch}{Probleemstelling}{Problem Statement}}
\label{sec:probleemstelling}

De probleemstelling in zijn simpelste vorm kan herleid worden naar:\\
\textit{'Kan PowerApps gebruikt worden om LanReview te vervangen?'}

De meerwaarde die het onderzoek (het antwoord op deze vraag) biedt aan AZ Glorieux:\\
De proof-of-concept kan een startpunt zijn voor de definitieve oplossing. Op z'n minst zal duidelijk worden hoe PowerApps in te zetten is en voor welke scenario's. Dit kan met andere woorden helpen bij de vernieuwing van andere applicaties en het kan de drempel verlagen voor het praktisch gebruik van PowerApps in scenario's waar men anders niet aan gedacht zou hebben.

\section{\IfLanguageName{dutch}{Onderzoeksvraag}{Research question}}
\label{sec:onderzoeksvraag}

\subsection{Hoofdonderzoeksvragen}

\begin{itemize}
    \item Is het mogelijk een vervanger voor LanReview te bouwen met Power Apps die elk de vier hoofdfunctionaliteiten ondersteunt en op z'n minst drie vierde van de overige functionaliteiten kan ondersteunen?
    \item Is Power Apps werkelijk de beste keuze hiervoor of is er meerwaarde in een volledig gerealiseerd IT asset management pakket? Is er alternatief een beter geschikt low-code platform?
\end{itemize}

\subsection{Deelonderzoeksvragen}

\begin{itemize}[label={$\circ$}]
    \item Is de Proof of Concept eenvoudig uit te breiden met nieuwe functionaliteiten? Is dit aanvaardbaar voor het IT team van AZ Glorieux?
    \item Is Power Apps robuust genoeg om meer complexe functionaliteit te ondersteunen. Is het met een zelf geschreven uitbreiding bijvoorbeeld mogelijk om de remote desktop functionaliteit te verwezenlijken?
    \item Is de gebruikte methode op z'n minst deels bruikbaar om andere applicaties voor de IT van AZ Glorieux te bouwen? Er worden twee cases onderzocht: één voor een telefoonboek (legacy business applicatie met een lage moeilijkheidsgraad) en één om het potentieel van Power Apps te demonstreren. Dit is opgenomen in de volgende onderzoeksvraag.
    \item Is er een use case voor nieuwe of experimentele functionaliteiten in Power Apps zoals integratie met Teams of beperkt toepassen van AI via de AI Builder?
    \item De Proof of Concept zal nauw samenwerken met SCCM, is het concreet mogelijk om de nabije toekomst ook samen te werken met Intune?
\end{itemize}

\section{\IfLanguageName{dutch}{Onderzoeksdoelstelling}{Research objective}}
\label{sec:onderzoeksdoelstelling}

De hoofddoelstelling bestaat uit het opstellen van een competente Proof of Concept. Hoe competent deze is wordt bepaald door de voldoening aan de twee hoofdonderzoeksvragen.\\
Hiernaast moet verdere ontwikkeling zo toegankelijk mogelijk gemaakt worden. Met andere woorden de proef en bijbehorende documenten moeten een hoge mate van praktisch nut hebben voor het IT Team om niet gerealiseerde functionaliteit in de proof-of-concept en toekomstige functionaliteiten toe te kunnen voegen.


\section{\IfLanguageName{dutch}{Opzet van deze bachelorproef}{Structure of this bachelor thesis}}
\label{sec:opzet-bachelorproef}

% Het is gebruikelijk aan het einde van de inleiding een overzicht te
% geven van de opbouw van de rest van de tekst. Deze sectie bevat al een aanzet
% die je kan aanvullen/aanpassen in functie van je eigen tekst.

De rest van deze bachelorproef is als volgt opgebouwd:

In Hoofdstuk~\ref{ch:stand-van-zaken} wordt een overzicht gegeven van de stand van zaken binnen het onderzoeksdomein, op basis van een literatuurstudie.

In Hoofdstuk~\ref{ch:methodologie} wordt de methodologie toegelicht en worden de gebruikte onderzoekstechnieken besproken om een antwoord te kunnen formuleren op de onderzoeksvragen.

In Hoofdstuk~\ref{ch:powerapps-poc} wordt de primaire proof-of-concept, gebouwd met PowerApps, overlopen.

Hoofdstuk~\ref{ch:outsystems-poc} behandelt de secundaire proof-of-concept, gemaakt met Outsystems.          

In Hoofdstuk~\ref{ch:conclusie}, tenslotte, wordt de conclusie gegeven en een antwoord geformuleerd op de onderzoeksvragen. Daarbij wordt ook een aanzet gegeven voor toekomstig onderzoek binnen dit domein.