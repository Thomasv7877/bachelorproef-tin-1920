%%=============================================================================
%% POC Outsystems
%%=============================================================================

\chapter{POC: Outsystems}
\label{ch:outsystems-poc}

%% TODO: Hoe ben je te werk gegaan? Verdeel je onderzoek in grote fasen, en
%% licht in elke fase toe welke stappen je gevolgd hebt. Verantwoord waarom je
%% op deze manier te werk gegaan bent. Je moet kunnen aantonen dat je de best
%% mogelijke manier toegepast hebt om een antwoord te vinden op de
%% onderzoeksvraag.

% TODO: kleine inleiding

\section{Voorbereiding}

\subsection{Van 0 tot de eerste app uitvoering}

De eerste stap is om een account aan te maken, hierna kan de Outsystems software geinstalleerd worden (Service Studio als gewone IDE en Extension studio om uitbreidingen te maken).\\
In de persoonlijke omgeving kan naast het maken van een nieuwe app community componenten van de Forge ook rechtstreeks geinstalleerd worden. \\
Bij het maken van een nieuwe app zijn er vijf keuzes: reactive of tradiotionele wab app. Tablet of Telefoon app. Ten laatste ook een service. Er is voor reactive gekozen omdat er in de POC in PowerApps twee versies gemaakt zijn en de app dus over meerde platformen moet kunnen gebruikt worden. Na de creatie van de app wordt ook een module aangemaakt.\\
De eenvoudigste manier om data te introduceren is door een Excel file op te laden. De data hieruit wordt naar een cloud instantie van SQL Server opgeslagen.\\
Als de app gestart wordt moeten Outsystems credentials opgegeven worden.
Dit is de basis om de requirements uit te kunnen beginnen werken maar eerst nog wat uitleg bij het gebruik van de IDE.

\subsection{IDE en begrippen}
Er zijn een aantal functies aanwezig, hier uitgelegd per mogelijke weergave paneel:
\begin{itemize}
    \item \textbf{Centraal:} Hier kunnen de UI flows (onderlinge verhouding van de schermen), ontwerpweergave van een scherm of een methode weegave staan.
    \item \textbf{Links:} Een overzicht van de beschikbare UI controls of methode acties.
    \item \textbf{Rechts:} Context views voor de processen, interface, de logica en de data. Aangevuld met een properties paneel wanneer toepasselijk.
\end{itemize}

[afb-ide]

Er zijn een aantal begrippen die gekend moeten zijn bij het maken van applicaties.
\begin{itemize}
    \item TrueChange Debugger:\\
    Zoals verwacht van een debugger kunnen breakpoints gezet worden en zijn tijdnes uitvoering de waarden van de actieve variabelen in te kijken. Het interessante is dat hiernaast de app geanalyseerd wordt en performantie en security aanbevelingen teruggegeven worden.
    \item One Click Publish:\\
    todo
\end{itemize}

\section{Model van Opstelling}



\section{Requirements}

