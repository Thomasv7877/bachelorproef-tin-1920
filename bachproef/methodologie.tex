%%=============================================================================
%% Methodologie
%%=============================================================================

\chapter{\IfLanguageName{dutch}{Methodologie}{Methodology}}
\label{ch:methodologie}

%% TODO: Hoe ben je te werk gegaan? Verdeel je onderzoek in grote fasen, en
%% licht in elke fase toe welke stappen je gevolgd hebt. Verantwoord waarom je
%% op deze manier te werk gegaan bent. Je moet kunnen aantonen dat je de best
%% mogelijke manier toegepast hebt om een antwoord te vinden op de
%% onderzoeksvraag.

%\lipsum[21-25]

\section{Requirementsanalyse}

\subsection{Functioneel VS Niet-functioneel}

\begin{itemize}
    \item \textbf{Functionele requirements}
    \begin{itemize}
        \item req 1
        \item req 2
    \end{itemize}
    \item \textbf{Niet-functionele requirements}
    \begin{itemize}
        \item req 1
        \item req 2
    \end{itemize}
\end{itemize}

\subsection{MoSCoW-methode}

\begin{itemize}
    \item \textbf{\colorbox{green}{Must have}}
    \begin{itemize}
        \item 
    \end{itemize}
    \item \textbf{\colorbox{yellow}{Should have}}
    \begin{itemize}
        \item 
    \end{itemize}
    \item \textbf{\colorbox{orange}{Could have (nice to have)}}
    \begin{itemize}
        \item 
    \end{itemize}
    \item \textbf{\colorbox{red}{Won't have}}
    \begin{itemize}
        \item 
    \end{itemize}
\end{itemize}

\subsection{Long List}

% TODO: tabel

\subsection{Short List}

% TODO

