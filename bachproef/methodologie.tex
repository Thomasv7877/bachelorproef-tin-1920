%%=============================================================================
%% Methodologie
%%=============================================================================

\chapter{\IfLanguageName{dutch}{Methodologie}{Methodology}}
\label{ch:methodologie}

%% TODO: Hoe ben je te werk gegaan? Verdeel je onderzoek in grote fasen, en
%% licht in elke fase toe welke stappen je gevolgd hebt. Verantwoord waarom je
%% op deze manier te werk gegaan bent. Je moet kunnen aantonen dat je de best
%% mogelijke manier toegepast hebt om een antwoord te vinden op de
%% onderzoeksvraag.

%\lipsum[21-25]

\section{Requirementsanalyse}

\subsection{Functioneel VS Niet-functioneel}

\begin{itemize}
    \item \textbf{Functionele requirements}
    \begin{itemize}
        \item Overzicht kunnen geven van belangrijkste info voor elk toestel in het netwerk.
        \item Rapporten kunnen genereren.
        \item PC's bedienen vanop afstand (Remote Desktop Protocol kunnen oproepen).
        \item Gerichte/basis taken kunnen automatiseren.
        \item Bruikbaar zijn buiten het domein.
        \item Mobiel bruikbaar zijn.
        \item Nieuwe types toestellen opnemen.
        \item Randapparatuur opnemen (= koppeling tussen toestellen).
        \item Revisie van elk dataveld per toesteltype.
        \item Data opslaan in SharePoint (cloud).
        \item Samenwerken met SCCM en/of synchroniseren met en data uit de SQL databank kunnen gebruiken.
        \item Barcode's kunnen scannen (toegang hebben tot camera, barcode functionaliteit ingebouwd).
        \item AI functionaliteit.
        \item Het Ping commando kunnen oproepen.
        \item Command line toegang hebben tot PC's.
        \item Intune integratie.
    \end{itemize}
    \item \textbf{Niet-functionele requirements}
    \begin{itemize}
        \item Prijs (geen prijsstijging voor de te implementeren case of voor het aantal gebruikers).
        \item Future proof\footnote{Hiermee bedoelt een algemene combinatie van achterliggende technologie, innovatie, consistentie en marktaanwezigheid} zijn.
        \item Diverse GUI verbeteringen (specifiek kleur markeringen, tabbladen)/robuust GUI ontwerp ondersteunen.
        \item Performant zijn (specifiek met grote hoeveelheden entries/rijen kunnen omgaan).
        \item Leercurve moet degelijk zijn (het gebruik ervan moet aanslaan na het onderzoek).
        \item Veiligheid (een must nodig omwille van gevoelige bedrijfsdata).
    \end{itemize}
\end{itemize}

\subsection{MoSCoW-methode}

\begin{itemize}
    \item \textbf{\colorbox{green}{Must have}}
    \begin{itemize}
        \item Prijs.
        \item Overzicht kunnen geven van belangrijkste info voor elk toestel in het netwerk.
        \item Rapporten kunnen genereren.
        \item PC's bedienen vanop afstand (Remote Desktop Protocol kunnen oproepen).
        \item Mobiel bruikbaar zijn.
        \item Future proof zijn.
        \item Performant zijn.
        \item Veiligheid.
        
    \end{itemize}
    \item \textbf{\colorbox{yellow}{Should have}}
    \begin{itemize}
        \item Gerichte/basis taken kunnen automatiseren.
        \item Bruikbaar zijn buiten het domein.
        \item Nieuwe types toestellen opnemen.
        \item Randapparatuur opnemen.
        \item Revisie van elk dataveld per toesteltype.
        \item Data opslaan in SharePoint (cloud).
        \item Leercurve moet degelijk zijn.
    \end{itemize}
    \item \textbf{\colorbox{orange}{Could have (nice to have)}}
    \begin{itemize}
        \item Samenwerken met SCCM en/of synchroniseren met en data uit de SQL databank kunnen gebruiken.
        \item Diverse GUI verbeteringen/robuust GUI ontwerp ondersteunen.
        \item Barcode's kunnen scannen.
        \item AI functionaliteit.
        \item Het Ping commando kunnen oproepen.
    \end{itemize}
    \item \textbf{\colorbox{red}{Won't have}}
    \begin{itemize}
        \item Command line toegang hebben tot PC's.
        \item Intune integratie.
    \end{itemize}
\end{itemize}

\subsection{Long List}

% Please add the following required packages to your document preamble:
% \usepackage{longtable}
% Note: It may be necessary to compile the document several times to get a multi-page table to line up properly
\begin{longtable}{llccc}
    \textbf{} & \textbf{} & \multicolumn{2}{c}{\textbf{Leider}} & \multicolumn{1}{l}{\textbf{}} \\
    \endfirsthead
    %
    \endhead
    %
    \textbf{Naam} & \textbf{Omschrijving} & \multicolumn{1}{l}{\textbf{Forrester}} & \multicolumn{1}{l}{\textbf{Gartner}} & \multicolumn{1}{l}{\textbf{Freemium}} \\
    AgilePoint &  &  &  &  \\
    Appian &  &  & x & x \\
    Betty Blocks &  &  &  &  \\
    bpm'online &  &  &  &  \\
    Clear Software &  &  &  &  \\
    GeneXus &  &  &  &  \\
    Kony &  & x &  & x \\
    K2 &  &  &  &  \\
    Kintone &  &  &  &  \\
    MatSoft &  &  &  & x \\
    Mendix &  & x & x & x \\
    Microsoft &  & x & x & x \\
    Outsystems &  & x & x & x \\
    Oracle &  &  &  &  \\
    Pega &  &  &  &  \\
    Progress Software &  &  &  &  \\
    ProntoForms &  &  &  &  \\
    Quick Base &  &  &  &  \\
    Salesforce &  & x & x & x \\
    ServiceNow &  &  &  & x \\
    Skuid &  &  &  &  \\
    Thinkwise &  &  &  &  \\
    TrackVia &  &  &  & x \\
    WaveMaker &  &  &  &  \\
    Zoho &  &  &  & x
\end{longtable}

\subsection{Short List}

% TODO

\subsubsection{Microsoft}

\subsubsection{Outsystems}

\subsubsection{Mendix}

\subsubsection{Salesforce}

