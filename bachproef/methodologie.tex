%%=============================================================================
%% Methodologie
%%=============================================================================

\chapter{\IfLanguageName{dutch}{Methodologie}{Methodology}}
\label{ch:methodologie}

%% TODO: Hoe ben je te werk gegaan? Verdeel je onderzoek in grote fasen, en
%% licht in elke fase toe welke stappen je gevolgd hebt. Verantwoord waarom je
%% op deze manier te werk gegaan bent. Je moet kunnen aantonen dat je de best
%% mogelijke manier toegepast hebt om een antwoord te vinden op de
%% onderzoeksvraag.

%\lipsum[21-25]

\section{Requirementsanalyse}


De bedoeling van deze analyse is uitvinden hoe goed Power Apps voldoet aan de requirements en mogelijks ontdekken of er een alternatief bestaat dat overwogen kan worden. Als PowerApps in de toekomst effectief gebruikt zal worden, voornamelijk door het IT en Helpdesk team, is het dan niet logisch om dit alternatief te laten conformeren aan de bedoelde eindgebruiker, zijnde professionele IT'ers? Dit idee zal mee spelen bij de besluitvorming.

\subsection{Functioneel VS Niet-functioneel}
De requirements werden opgesteld tijdens herhaalde gesprekken met zowel de co-promotor als een lid van het Helpdesk team dat de app effectief zou gaan gebruiken.
De resultaten zijn hieronder opgesplitst in functionele-\footnote{Functionele requirement: gewenst gedrag ven het systeem} en niet-functionele\footnote{Niet-functionele requirement: kwaliteitseis waaraan het systeem moet voldoen} requirements.

\begin{itemize}
    \item \textbf{Functionele requirements}
    \begin{itemize}
        \item Overzicht kunnen geven van belangrijkste info voor elk toestel in het netwerk.
        \item Rapporten kunnen genereren.
        \item PC's bedienen vanop afstand (Remote Desktop Protocol kunnen oproepen).
        \item Gerichte/basis taken kunnen automatiseren.
        \item Bruikbaar zijn buiten het domein.
        \item Mobiel bruikbaar zijn.
        \item Nieuwe types toestellen opnemen.
        \item Randapparatuur opnemen (= koppeling tussen toestellen).
        \item Revisie van elk dataveld per toesteltype.
        \item Data opslaan in SharePoint (cloud).
        \item Samenwerken met SCCM en/of synchroniseren met en data uit de SQL databank kunnen gebruiken.
        \item Barcodes kunnen scannen (toegang hebben tot camera, barcode functionaliteit ingebouwd).
        \item AI functionaliteit.
        \item Het Ping commando kunnen oproepen.
        \item Command line toegang hebben tot PC's.
        \item Intune integratie.
    \end{itemize}
    \item \textbf{Niet-functionele requirements}
    \begin{itemize}
        \item Prijs (geen prijsstijging voor de te implementeren case of voor het aantal gebruikers).
        \item Future proof\footnote{Hiermee bedoelt een algemene combinatie van achterliggende technologie, innovatie, consistentie en marktaanwezigheid} zijn.
        \item Diverse GUI verbeteringen (specifiek kleur markeringen, tabbladen)/robuust GUI ontwerp ondersteunen.
        \item Performant zijn (specifiek met grote hoeveelheden entries/rijen kunnen omgaan).
        \item Leercurve moet degelijk zijn (het gebruik ervan moet aanslaan na het onderzoek).
        \item Veiligheid (een must, nodig omwille van gevoelige bedrijfsdata).
    \end{itemize}
\end{itemize}

\subsection{MoSCoW-methode}

De volgende stap is prioriteiten stellen onder de requirements. Hiervoor wordt de MoSCoW-methode\footnote{Must have: eis moet terugkomen in het eindresultaat, Should have: eis is zeer gewenst maar het product is bruikbaar zonder, Could have: eis komt aan bod als er genoeg tijd is, Won't have: komt niet aan bod, is voor de toekomst of een vervolgproject \autocite{Wikipedia2020}} gebruikt.

\begin{itemize}
    \item \textbf{\colorbox{green}{Must have}}
    \begin{itemize}
        \item Geen of beperkte meerprijs.
        \item Overzicht kunnen geven van belangrijkste info voor elk toestel in het netwerk.
        \item Rapporten kunnen genereren.
        \item PC's bedienen vanop afstand (Remote Desktop Protocol kunnen oproepen).
        \item Mobiel bruikbaar zijn.
        \item Future proof zijn.
        \item Performant zijn.
        \item Veiligheid.
        
    \end{itemize}
    \item \textbf{\colorbox{yellow}{Should have}}
    \begin{itemize}
        \item Gerichte/basis taken kunnen automatiseren.\\
        \textit{(Specifieke case: Een wekelijks rapport opstellen en rondmailen.)}
        \item Bruikbaar zijn buiten het domein.
        \item Nieuwe types toestellen opnemen.
        \item Randapparatuur opnemen.
        \item Revisie van elk dataveld per toesteltype.
        \item Data opslaan in SharePoint (cloud).
        \item Leercurve moet degelijk zijn.
    \end{itemize}
    \item \textbf{\colorbox{orange}{Could have (nice to have)}}
    \begin{itemize}
        \item Samenwerken met SCCM en/of synchroniseren met en data uit de SQL databank kunnen gebruiken.
        \item Diverse GUI verbeteringen/robuust GUI ontwerp ondersteunen.
        \item Barcodes kunnen scannen.
        \item AI functionaliteit.
        \item Het Ping commando kunnen oproepen.
    \end{itemize}
    \item \textbf{\colorbox{red}{Won't have}}
    \begin{itemize}
        \item Command line toegang hebben tot PC's.
        \item Intune integratie.
    \end{itemize}
\end{itemize}

\subsection{Long List}

Er zijn veel platformen om uit te kiezen\footnote{ 57 volgens een oplijsting van \href{https://www.trustradius.com/low-code-development}{TrustRadius}}. Degene hieronder opgelijst werden besproken in de Forrester Wave \autocite{Rymer2019} en het Gartner Magic Quadrant \autocite{Vincent2019}. De low-code markt kent een sterke groei en is zeer bewogen (zie sectie~\ref{sec:markt-en-evolutie}). Het eerste gekozen criteria om de lijst te filteren is bedoelt om hiermee om te gaan, het kan immers aangenomen worden dat een platform bestempelt door Gartner of Forrester als leider voldoende future proof is. Het andere gekozen criteria betreft het licentie model. Power Apps is opgenomen in het Office 365 licentie van AZ Glorieux, initieel gebruik ervan betekend geen dus geen meerprijs, een concurrent moet met andere woorden iets gelijkaardig kunnen aanbieden.

\begin{longtable}{llccc}
    \textbf{} & \textbf{} & \multicolumn{2}{c}{\textbf{Leider}} & \multicolumn{1}{l}{\textbf{}} \\
    \endfirsthead
    %
    \endhead
    %
    \textbf{Naam} &  & \multicolumn{1}{l}{\textbf{Forrester}} & \multicolumn{1}{l}{\textbf{Gartner}} & \multicolumn{1}{l}{\textbf{Freemium}} \\
    AgilePoint &  &  &  &  \\
    Appian &  &  & x &  \\
    Betty Blocks &  &  &  &  \\
    bpm'online &  &  &  &  \\
    Clear Software &  &  &  &  \\
    GeneXus &  &  &  &  \\
    Kony &  & x &  & x \\
    K2 &  &  &  &  \\
    Kintone &  &  &  &  \\
    MatSoft &  &  &  & x \\
    Mendix &  & x & x & x \\
    Microsoft &  & x & x & (x) \\
    Outsystems &  & x & x & x \\
    Oracle &  &  &  &  \\
    Pega &  &  &  &  \\
    Progress Software &  &  &  &  \\
    ProntoForms &  &  &  &  \\
    Quick Base &  &  &  &  \\
    Salesforce &  & x & x &  \\
    ServiceNow &  &  &  & x \\
    Skuid &  &  &  &  \\
    Thinkwise &  &  &  &  \\
    TrackVia &  &  &  & x \\
    WaveMaker &  &  &  &  \\
    Zoho &  &  &  & x
\end{longtable}

Het resultaat is dat er van de leiders (zie tabel~\ref{table:leiders} voor een gefocuste weergave) slechts enkelen overblijven die in aanmerking komen.

\subsection{Short List}

% TODO: uitleg herleiden van requirements beter nuanceren?

Hier worden de overgebleven platformen in diepte met elkaar vergeleken Maar in plaats van elke requirement af te toetsen werden deze eerst herleid naar hun achterliggende begrippen (bijvoorbeeld: mobile ondersteuning, automatisatie functionaliteit, security, leercurve en dergelijke).

Gemaakte stellingen zijn afkomstig uit PCmagazine.com reviews van elk platform en zijn aangevuld met informatie uit de officiële documentatie per platvorm.
% TODO: pcMag veotnoot met overzicht reviews

\subsubsection{Microsoft}

Power Apps is deel van het Power platform (zie afbeelding~\ref{fig:mspowerplatform}). Het bestaat sinds 2016 en is daarmee het jongste platform in de vergelijking maar heeft een snelle ontwikkeling ondergaan en is op slechts enkele jaren tijd een marktleider geworden.

Werd diepgaand behandeld in de Stand van Zaken. (specifiek in sectie~\ref{sec:power-platform}). Een kort overzicht:

Gefocust op business users.
Apps worden gebouwd met de Power Apps Studio in de cloud. Deze omgeving is volledig grafisch, er is geen code editing. Uitbreidingen kunnen enkel ingevoegd worden via Azure Functions of Azure web apps en worden geschreven in .NET. Logica is gedeclareerd met een Excel-achtige formule taal. Er is een grote catalogus aan connectors naar externe service of data opslag. Verbinding maken met SharePoint data is eenvoudig met de Connector ervoor. Automatisatie is mogelijk via Power Automate waarin geautomatiseerde stromen van aaneengeschakelde acties geconfigureerd kunnen worden. AI werd recent geïntroduceerd, dit component heet AI Builder, naast zelf trainen van gangbare modellen biedt Microsoft enkele voorgetrainde aan. Nog in bèta is het testing framework genaamd UI Test Studio. Power Apps zijn gefocust op interne business gebruikers, voor externe bestaat er Portals. De leercurve is hoger dan van de concurrentie maar beschikbare documentatie en tutorials zijn talrijk en matuur. Performantie en vooral database performantie is lager dan van de concurrentie, er staat ook standaard een limiet op het aantal rijen dat per keer opgehaald kan worden (500). Microsoft toont toewijding aan het platform door constante ontwikkeling maar sommige licentiewijzigingen zijn minder goed ontvangen.
% TODO: security aanvullen + bron met ()kritiek op) licentiewijziging

\begin{itemize}
    \item \textbf{Forrester}
    \begin{itemize}
        \item \textit{Positief:} Maturiteit bereikt, krachtige features en grote catalogus aan integratie adapters.
        \item \textit{Negatief:} Verwarrend product aanbod en licentiëring. Voor elk product van het Power platform is bijvoorbeeld een aparte licentie nodig.
    \end{itemize}
    \item \textbf{Gartner}
    \begin{itemize}
        \item \textit{Positief:} Eenvoudige drag-en-drop design tool en Exel-achtige  expressie taal. Snelle productie deployment mogelijk. Sterke toewijding aan LCAP markt getoond.
        \item \textit{Negatief:} Model apps niet altijd beste oplossing. Verwarrende licentiëring.
    \end{itemize}
\end{itemize}

\begin{table}[h!] Licentiëring:  
\begin{longtable}{|l|c|c|c|c|}
    \hline
    & \multicolumn{1}{l|}{\textbf{PA for Office 365}} & \multicolumn{1}{l|}{\textbf{PA Plan 1}} & \multicolumn{1}{l|}{\textbf{PA plan 2}} & \multicolumn{1}{l|}{\textbf{PA for Dynamics 365}} \\ \hline
    \endfirsthead
    %
    \endhead
    %
    \textbf{Canvas apps} & x & x & x & x \\ \hline
    \textbf{Model driven apps} &  &  & x & x \\ \hline
    \textbf{\begin{tabular}[c]{@{}l@{}}Common Data Services\\ gebruik mogelijk\end{tabular}} &  & x & x & x \\ \hline
    \textbf{Standaard connectors} & x & x & x & x \\ \hline
    \textbf{Premium Connectors} &  & x & x & x \\ \hline
    \textbf{Custom connectors} &  & x & x & x \\ \hline
    \textbf{On-premises connectors} &  & x & x & x \\ \hline
\end{longtable}
\caption{Overzicht van beschikbare plannen \autocite{Pohl2019}}
\end{table}

\subsubsection{Outsystems}

Outsystems werd opgericht in Lissabon in 2001, zit momenteel in 11 landen en telt 1228 medewerkers.

Er is een focus op professionele ontwikkelaars, het is gemakkelijker om met code te werken. Het is mogelijk om op elk moment te switchen van de grafische omgeving naar de code editor. Iets anders dat dit professioneel aspect ondersteund is dat de IDE (genaamd Service Studio) offline geïnstalleerd dient te worden. De layout ervan komt sterk overeen met die van PowerApps en Mendix. Andere grote onderdelen zijn de Forge (respository voor gebruikersgemaakte apps en plug-ins) en de Integration Studio waar extensions geschreven worden.\\
Een nadeel van de technische focus is dat er geen volledig cloud no-code omgeving aanwezig is. Een ander nadeel is de hogere leercurve maar er is een uitgebreid aanbod van documentatie, tutorials online cursussen en webinars aanwezig in de OutSystems University om hiermee om te gaan.\\
Het platform is gebouwd in .NET. Bouwen van mobiele of webapplicaties is ondersteund maar er moet gekozen worden tussen de twee.\\
Een app is gestructureerd in modules: mobile (of web), service, library en extension modules. Dit om hergebruik aan te moedigen.\\
Automatisatie functionaliteit is beperkt tot Timers. Het is enkel mogelijk om bepaalde acties op een tijdschema uit te voeren. Wat AI betreft zijn er connectors naar Azure Luis en Azure ML. Inhuis voorziet het Outsystems.AI programma taal analyse componenten (sleutelzin detectie, gevoelsanalyse).\\
UI design is minder geavanceerd dan bij de concurrentie en er is meer werk nodig om het gewenst resultaat te krijgen.\\
SharePoint gebruik is enkel mogelijk via de REST API. Dit is een grote beperking ten opzichte van Power Apps, waar het een standaard Connector is.\\
Het platform is veilig en voldoet aan volgende certifieeringen: % TODO
Om gerust te stellen voor vendor lock-in\footnote{Gedwongen een product of service moeten blijven gebruiken, onafhankelijk van kwaliteit, omdat het (financieel) niet praktisch is om er vanaf te stappen \autocite{Cloudflare}} is het mogelijk om de applicatie te genereren in .NET.\\
\autocite{Marvin2017} \textit{en aangevuld met informatie uit de officiële documentatie\footnote{\url{https://success.outsystems.com/Documentation}}} 

\begin{itemize}
    \item \textbf{Forrester}
    \begin{itemize}
        \item \textit{Positief:} Krachtigste aanbod van features en consistent met introduceren van nieuwe features. Globale aanwezigheid.
        \item \textit{Negatief:} Soms manuele code nodig voor integratie. Complex prijsmodel dat potentiële klanten kan afschrikken.
    \end{itemize}
    \item \textbf{Gartner}
    \begin{itemize}
        \item \textit{Positief:} Sterke visie en innovatie. Gedeelde Forge componenten hoog beoordeelt. Bovengemiddeld op gebied van productiviteit en modernisatie van bestaande applicaties.
        \item \textit{Negatief:} Niet competitief voor bouwen van proces georiënteerde apps. Minder toegankelijk voor citizen developers. Prijzen kunnen snel stijgen afhankelijk van gebruik, dit gebruik wordt op verwarrende manier berekend.
    \end{itemize}
\end{itemize}

\begin{table}[h!] Licentiëring: 
\begin{longtable}{|l|c|c|c|c|}
    \hline
    & \multicolumn{1}{l|}{\textbf{Free}} & \multicolumn{1}{l|}{\textbf{Basic}} & \multicolumn{1}{l|}{\textbf{Standard}} & \multicolumn{1}{l|}{\textbf{Enterprise}} \\ \hline
    \endfirsthead
    %
    \endhead
    %
    \textbf{Unlimited apps} & (x)\footnote{Gedeelde infrastructuur en database gelimiteerd tot 2GB} & x & x & x \\ \hline
    \textbf{Gebruikers} & 100 & 1000 & Geen limiet & Geen limiet \\ \hline
    \textbf{\begin{tabular}[c]{@{}l@{}}On-premises of \\ \\ private cloud\end{tabular}} &  &  & x & x \\ \hline
    \textbf{Support} &  & 8x5 & 8x5 & 24/7 \\ \hline
    \textbf{Aantal omgevingen} & 1 & 3 & 3 & 5 \\ \hline
    \textbf{CI \& deployment} &  & x & x & x \\ \hline
    \textbf{} &  &  &  &  \\ \hline
    \textbf{Maandelijkse prijs} & Gratis & \textgreater \$4000 & \textgreater{}\$10000 & Custom \\ \hline
\end{longtable}
\caption{Prijsmodel Outsystems \autocite{Outsystems}}
\label{tab:price-outsystems}
\end{table}

\subsubsection{Mendix}

Mendix werd opgericht in Nederland in 2005, werd in 2018 overgenomen door Siemens voor 730 miljoen dollar. Er zijn enkele grote partnerships aangegaan, onder andere met SAP die Mendix resellt als het SQL Cloud Platform.

Net als bij Outsystems is de doelgroep eerder professionele ontwikkelaars. Er is zowel een volledige no-code cloud omgeving aanwezig (Mendix Studio) als een uitgebreidere variant (Mendix Studio Pro) die offline geïnstalleerd moet worden, eerder gericht op developers, aan wie het dan mogelijk gemaakt wordt om te coderen in Java.\\
Er is een groot aanbod aan voorgebouwde templates en componenten die Microsoft en Outsystems evenaren.
De UI filosofie is om te starten met design en wireframes, dan het model te maken met logica en workflows die in dat model passen. De apps zijn responsief. Het scherm wordt automatisch aangepast tussen smartphone, tablet of desktop views.\\
Database integratie is beter dan bij Outsystems. Database wijzigingen worden automatisch herkend in de app, bij Outsystems is dit niet het geval. Specifiek naar SharePoint toe is verbinding net als bij Outsystems beperkt tot een REST API.\\
App creatie is algemeen gezien gestroomlijnder dan bij Outsystems. Er kan na afloop gedeployed worden naar verschillende cloud omgevingen. % TODO: info over containerisatie
App aanpassingen zijn eenvoudig (versionering is voorzien) maar voor elke nieuwe versie van het platform moet er gemigreerd worden.\\
Testing en analystics zijn geavanceerder dan bij concurrenten zoals bijvoorbeeld PowerApps waar dit nog in bèta zit. Bij Mendix is een hoge mate van maturiteit bereikt, tests worden bijvoorbeeld automatsch uitgevoerd.\\
Over specifieke onderdelen: Buzz is een portaal dat dient als sociaal intranet waar collaboratie begint. SCRUM is ingebouwd in het platform. Integraties, plugins, gebruikersapps zijn in de Mendix App Store te vinden, dit aspect is even matuur als varianten van Microsoft en Outsystems.\\
Wat AI functionaliteit betreft is er integratie mogelijk met IBM Watson.\\
Security is als verwacht voor een marktleider: \\ % TODO
\autocite{Marvin2017a} \textit{en aangevuld met informatie uit de officiële documentatie\footnote{\url{https://docs.mendix.com/}}}


\begin{itemize}
    \item \textbf{Forrester}
    \begin{itemize}
        \item \textit{Positief:} Trendsetter in features, krachtig aanbod. Goede ondersteuning voor app levenscyclus, uitbreidingen naar CI toe. Sterke partners: SAP, Siemens en IBM.
        \item \textit{Negatief:} Extra code nodig voor integraties. Services voor content management binnen apps is iets zwakker. Prijzen voor platform adoptie moeilijk te voorspellen.
    \end{itemize}
    \item \textbf{Gartner}
    \begin{itemize}
        \item \textit{Positief:} Competitief sterk door invloed van middelen na overname door Siemens. Aantrekkelijk voor enterprise door autoscaling, high availability en lage latency. Aparte ontwikkelomgeving afhankelijk van bedoede gebruiker. AI ondersteunde ontwikkeling. Mogelijk complexe applicaties te maken. Tevredenheid van professionele ontwikkelaars.
        \item \textit{Negatief:} Na overname door Siemens zou toewijding aan Mendix kunnen veranderen weg van mainstream LCAP. Minder bruikbaar door citizen developers. Prijzen en contract flexibiliteit slecht beoordeelt.
    \end{itemize}
\end{itemize}

 

\begin{table}[h!]Licentiëring: 
    \begin{longtable}{|l|c|c|c|c|}
        \hline
        \textbf{} & \multicolumn{1}{l|}{\textbf{Free edition}} & \multicolumn{1}{l|}{\textbf{Single app}} & \multicolumn{1}{l|}{\textbf{Professional}} & \multicolumn{1}{l|}{\textbf{Enterprise}} \\ \hline
        \endfirsthead
        %
        \endhead
        %
        \textbf{\begin{tabular}[c]{@{}l@{}}Number of\\ environments\end{tabular}} & 1 & 2 & 2 & 3 \\ \hline
        \textbf{Horizontal scaling} &  &  &  & x \\ \hline
        \textbf{\begin{tabular}[c]{@{}l@{}}Support for CI \& \\ deployment\end{tabular}} &  &  &  & x \\ \hline
        \textbf{On-premises} &  &  &  & x \\ \hline
        \textbf{App user limit} & unlimited & unlimited & unlimited & unlimited \\ \hline
        \textbf{Number of apps} & unlimited\footnote{1 GB geheugen en 0.5 GB geheugen per app} & unlimited & unlimited & unlimited \\ \hline
        & \multicolumn{1}{l|}{} & \multicolumn{1}{l|}{} & \multicolumn{1}{l|}{} & \multicolumn{1}{l|}{} \\ \hline
        \textbf{Price} & Free & \textgreater \$1917 & custom & custom \\ \hline
    \end{longtable}
    \caption{Prijsmodel Mendix \autocite{Mendix}}
    \label{tab:price-mendix}
\end{table}

\subsubsection{Salesforce}

Salesforce is een Amerikaans cloud-georiënteerd software bedrijf met hoofdzetel in San Francisco, Californië. Het verleent sinds 1999 CRM\footnote{\url{https://nl.wikipedia.org/wiki/Customer_relationship_management}} services en complementaire enterprise applicaties waaronder het Lightning low-code platform.

Het platform bied opties aan voor zowel low-code als traditionele ontwikkeling.\\
Er is een focus op CRM en het doelpubliek van de apps is business gebruikers. Voor externe gebruikers bestaat er de Lightning Community Builder. Dit is analoog met Microsoft zijn PowerApps en Portals.\\
De apps kunnen enkel vanuit het Salesforce portaal gebruikt worden. Hosting gebeurt altijd in de Salesforce cloud. Apps zijn standaard responsief.\\
De Lightning App Builder is een point-en-klik low-code oplossing. Ondersteunend zijn er de Process Builder om datamodellen op te stellen en de Schema Builder voor complexe geautomatiseerde logica. In de Schema Builder is gebruik van de binnenhuis ontwikkelde programmeertaal Apex ook mogelijk. De syntax ervan komt overeen met Java. Vanuit Apex kan data gemanipuleerd worden aan de hand van SOQL (Salesforce Object Query Language).\\
De UI wordt gebouwd met componenten uit het Lightning UI framework. Er is een sterke focus op herbruikbaarheid.\\
Er kunnen onder andere REST en SOAP API's gebruikt worden maar dit moet ingesteld worden via Apex.\\
Net als bij Outsystems kan app bouw geassisteerd worden door AI. Salesforce doet dit met Einstein. Hierbuiten kan Einstein ook gebruikt worden voor traditionele toepassingen in dit domein zoals AI assistentie en image herkenning met de Vision API. Chatbots bouwen is mogelijk met Einstein Bots.\\
Voor professionele ontwikkelaars is er het Heroku web development platform. Hier is men vrij om te kiezen welke programmeertaal gebruikt wordt.

Het heeft de meest uitgebreide feature-set en de beste aanbieding van community componenten en apps in de App Exchange. De keerzijde hiervan is dat de features overweldigend kunnen worden (feature bloat, UI clutter).\\
Er zijn trails (tutorials) beschikbaar op de trailhead interactieve leer-en training site maar de pagina's zijn niet altijd up-to-date.\\
Het is mogelijk dat database aanpassingen problemen geven in de app.
\autocite{Marvin2017b}

Salesforce biedt een gratis Developer Edition aan. Dit heeft alle features en is bedoelt om te leren. Er is ook een proefversie mogelijk. De betalende opties zijn te zien in Tabel~\ref{tab:plan-salesforce}

\begin{itemize}
    \item \textbf{Forrester}
    \begin{itemize}
        \item \textit{Positief:} Grote partner lijst en security certificaten.
        \item \textit{Negatief:} Code moeten gebruiken om aan noden te kunnen voldoen.
    \end{itemize}
    \item \textbf{Gartner}
    \begin{itemize}
        \item \textit{Positief:} App Exchange, innovatie (onder andere Einstein en IoT mogelijkheden)
        \item \textit{Negatief:} Zowel Lightning als Apex moeten gebruiken om requirements uit te kunnen werken.
    \end{itemize}
\end{itemize}

Licentiëring:
\begin{table}[h!]
    \centering
    \begin{tabular}{|l|c|c|}
        \hline
        \textbf{}                      & \multicolumn{1}{l|}{\textbf{Platform Starter}} & \multicolumn{1}{l|}{\textbf{Platform Plus}} \\ \hline
        \textbf{Custom objecten}       & 10                                             & 110                                         \\ \hline
        \textbf{Lightning App Builder} & x                                              & x                                           \\ \hline
        \textbf{AppExchange}           & x                                              & x                                           \\ \hline
        \textbf{Process Automatie}     & x                                              & x                                           \\ \hline
        \textbf{Lightning Console}      & (x)\textsuperscript{\$25 extra}                                            & x                                           \\ \hline
        \textbf{}                      &                                                &                                             \\ \hline
        \textbf{Maandelijkse prijs}    & \$25                                           & \$100                                       \\ \hline
    \end{tabular}
    \caption{Prijsmodel Salesforce \autocite{Salesforcea}}
    \label{tab:plan-salesforce}
\end{table}

\subsubsection{Appian}

Net als Salesforce is dit een grote speler. Functioneel is het even sterk als Outsystems en Mendix maar helaas is er geen gratis licentie mogelijk.

\subsubsection{Temenos Quantum (Kony)}

Dit platform is door \textcite{Vincent2019} bestempelt als een leider en heeft een gratis licentie. Wat serieuze overweging tegenhoud is de recente overname (en naamswijziging).

\subsection{Besluit}

Er zijn geen redenen gevonden om weg te stappen van PowerApps als primaire kandidaat om de POC mee uit te werken ook al kan het schrijven van uitbreidingen omslachtig worden. Er zijn twee concurrenten gevonden: Outsystems en Mendix. Wat functionaliteit betreft zijn ze inwisselbaar. Beiden zijn gefocust op professionele ontwikkelaars en hebben aantrekkelijke gratis licenties. De grootste belemmering betreft het gebruik van SharePoint als data opslag. Waar het bij PowerApps standaard ondersteund is als Connector moet er in Mendix en Outsystems zijn geval een REST API gebruikt worden. Bij gebruik van één van beiden is het dus interessanter om een alternatieve opslagmethode te gebruiken.

% TODO: keuze outsystems betere gratis versie dan mandix nuanceren
Na overweging is voor Outsystems gekozen om de secundaire POC mee uit te werken. Wat doorslag gaf is de gratis versie die van Outsystems net iets beter is.\\
Bovendien vallen de resource noden van de POC volledig binnen de limieten van het plan, in Subsectie~\ref{subsec:os-prijs} staat dit uitgelicht.


