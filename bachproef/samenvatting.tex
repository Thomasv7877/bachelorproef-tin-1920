%%=============================================================================
%% Samenvatting
%%=============================================================================

% TODO: De "abstract" of samenvatting is een kernachtige (~ 1 blz. voor een
% thesis) synthese van het document.
%
% Deze aspecten moeten zeker aan bod komen:
% - Context: waarom is dit werk belangrijk?
% - Nood: waarom moest dit onderzocht worden?
% - Taak: wat heb je precies gedaan?
% - Object: wat staat in dit document geschreven?
% - Resultaat: wat was het resultaat?
% - Conclusie: wat is/zijn de belangrijkste conclusie(s)?
% - Perspectief: blijven er nog vragen open die in de toekomst nog kunnen
%    onderzocht worden? Wat is een mogelijk vervolg voor jouw onderzoek?
%
% LET OP! Een samenvatting is GEEN voorwoord!

%%---------- Nederlandse samenvatting -----------------------------------------
%
% TODO: Als je je bachelorproef in het Engels schrijft, moet je eerst een
% Nederlandse samenvatting invoegen. Haal daarvoor onderstaande code uit
% commentaar.
% Wie zijn bachelorproef in het Nederlands schrijft, kan dit negeren, de inhoud
% wordt niet in het document ingevoegd.

\IfLanguageName{english}{%
\selectlanguage{dutch}
\chapter*{Samenvatting}
\lipsum[1-4]
\selectlanguage{english}
}{}

%%---------- Samenvatting -----------------------------------------------------
% De samenvatting in de hoofdtaal van het document

\chapter*{\IfLanguageName{dutch}{Samenvatting}{Abstract}}

Het ziekenhuis AZ Glorieux zou graag PowerApps willen gebruiken om een oude business applicatie (LanReview, ITAM) te vervangen. Specifiek wordt bekeken hoe geschikt PowerApps is voor deze rol.\\
Dit onderzoek is relevant omdat het algemene zakelijke landschap gekenmerkt is door een ontoereikendheid aan de vraag om software het is het interessant om te zien of low-code zijn beloofde snellere softwareontwikkeling kan waarmaken. 

Eerst wordt de omgeving van AZ Glorieux verkend en wordt LanReview in detail besproken. Het begrip low-code wordt onderzocht, in het bijzonder wordt er een actueel beeld van geschetst. Het PowerApps platform wordt onder de loep genomen.

Er wordt een requirementsanalyse gemaakt. De low-code markt wordt onderzocht en met de requirements in het achterhoofd wordt er een low-code (voor powerusers) platform gekozen voor de primaire proof-of-concept en hierna een low-code (voor ontwikkelaars) platform voor de secundaire proof-of-concept. Dit zijn PowerApps en Outsystems geworden.

In de volgende delen worden de proof-of-concepts voor beiden uitgewerkt. Dit wordt gedaan door elke requirement te implementeren en te documenteren.

Tenslotte komt het besluit dat PowerApps niet de beste optie is voor deze case en wordt genuanceerd voor wat het wel en niet geschikt zou zijn. Dit wordt gedaan voor zowel PowerApps (no-code) als Outsystems (low-code).