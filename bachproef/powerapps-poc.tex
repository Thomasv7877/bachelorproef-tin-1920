%%=============================================================================
%% POC Power Apps
%%=============================================================================

\chapter{POC: Power Apps}
\label{ch:powerapps-poc}

%% TODO: Hoe ben je te werk gegaan? Verdeel je onderzoek in grote fasen, en
%% licht in elke fase toe welke stappen je gevolgd hebt. Verantwoord waarom je
%% op deze manier te werk gegaan bent. Je moet kunnen aantonen dat je de best
%% mogelijke manier toegepast hebt om een antwoord te vinden op de
%% onderzoeksvraag.

% TODO: kleine inleiding

\section{Voorbereiding}

De POC is gemaakt in de cloud omgeving van PowerApps met de Office365 licentie van HoGent en AZ Glorieux. De beperkte mogelijkheden van deze licenties werden omzeild door gebruik van het Community Plan dat alle functionaliteitsrestricties opheft in een persoonlijke ontwikkelomgeving. Concreet beperkt de Office365 licentie uitbreidingen en toegang tot lokale resources. Eigen geschreven logica is mogelijk via custom connectors, deze moeten gehost of op z'n minst gedeclareerd zijn in Microsoft Azure. Dankzij het Azure Student plan via HoGent werd de nodige cloud capaciteit verschaft.

Eens de nodige licenties aanwezig zijn is er toegang tot de Power Apps ontwikkelomgeving. Om een canvas app te maken zijn er verscheidene opties:
\begin{itemize}
    \item \textbf{Beginnen vanuit data:} Er wordt een gegevensconnector gekozen die de app data verschaft. Op basis van deze data wordt een app gegenereerd met 3 schermen: een overzichtscherm met lisjtweergave van de items, een detailscherm dat geopend wordt na klikken op een item uit het vorige scherm dat meer datavelden toont en een edit scherm, geopend vanuit het detailscherm waar men de datavelden kan aanpassen. Het detail- en edit scherm toont de data via een formulier control.
    \item \textbf{Blanco beginnen:} Een app ontwikkelen vanuit een letterlijk leeg canvas.
    \item \textbf{Sjabloon:} Er zijn sjablonen voorzien voor gangbare scenario's.
\end{itemize}

Er is ook keuze tussen telefoon en tablet layout. De huidige case leent zich tot eerste optie maar hierin is men beperkt tot de telefoon layout. Een tablet layout is beter geschikt voor de hoeveelheid data die getoont moet kunnen worden en de extra grafische controls nodig voor sommige requirements. Om deze reden werd voor een leeg canvas met tablet layout gekozen.

\subsection{Data en data weergave}



\section{SharePoint Configuratie}



\section{Model van Opstelling}



\section{Requirements}

