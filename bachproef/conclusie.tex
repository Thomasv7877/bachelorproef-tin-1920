%%=============================================================================
%% Conclusie
%%=============================================================================

\chapter{Conclusie}
\label{ch:conclusie}

% TODO: Trek een duidelijke conclusie, in de vorm van een antwoord op de
% onderzoeksvra(a)g(en). Wat was jouw bijdrage aan het onderzoeksdomein en
% hoe biedt dit meerwaarde aan het vakgebied/doelgroep? 
% Reflecteer kritisch over het resultaat. In Engelse teksten wordt deze sectie
% ``Discussion'' genoemd. Had je deze uitkomst verwacht? Zijn er zaken die nog
% niet duidelijk zijn?
% Heeft het onderzoek geleid tot nieuwe vragen die uitnodigen tot verder 
%onderzoek?

Dit onderzoek werd uitgewerkt om antwoord te kunnen geven op de twee hoofdonderzoeksvragen, deze waren:
\begin{itemize}
    \item \textit{Is het mogelijk een vervanger voor LanReview te bouwen met Power Apps die elk de vier hoofdfunctionaliteiten ondersteunt en op z'n minst drie vierde van de overige functionaliteiten kan ondersteunen?}
    \item \textit{Is Power Apps werkelijk de beste keuze hiervoor of is er meerwaarde in een volledig gerealiseerd IT asset management pakket? Is er alternatief een beter geschikt low-code platform?}
\end{itemize}

Hoewel sommige requirements eenvoudig uit te werken waren (barcode scanning, automatisatie) waren voor anderen workarounds nodig die na uitwerking nog steeds tekort schoten vergeleken wat met de originele LanReview mogelijk was (RDP/ping, filtering). Indien men tijdens de uitwerking van een app workarounds en custom connectors moet beginnen gebruiken is het daarom aangeraden om een andere weg uit te gaan. Door de nodige tijdsinvestering kan evengoed iets opgesteld worden in een traditionele ontwikkelomgeving en het sterkste voordeel van low-code is daarmee teniet gedaan.\\
Dit wil niet zeggen dat PowerApps geen meerwaarde bied voor deze case. Voor het maken van een stikt mobiele versie met een beperktere functionaliteitsset is het zeer geschikt. Bijvoorbeeld voor te een telefoonboek app zou het ook perfect zijn.

Voor het maken van een complexere business app is Outsystems te beste keuze. de leercurve is niet dermate hoger dan die van PowerApps en indien de app beperkte resources en gebruikersbestand nodig heeft moet bovendien geen betalend plan aangegaan worden. Dit is het geval voor de LanReview POC maar er zijn nog steeds beperkingen mogelijk door de aard van de applicatie (cloud). Als lokale resources nodig zijn (RDP/ping) is dit net als bij PowerApps een hekelpunt. Het besluit hierbij is dat als AZ Glorieux een complexe app nodig heeft Outsystems een goede optie is maar als vervanger van LanReview is het niet toereikend.

De meest passende oplossing voor het vernieuwen van de desktop versie van LanReview is om de broncode over te zetten naar een WPF of UWP .NET Core desktop applicatie of soortgelijk. 

Low-code (PowerApps) heeft wel degelijk een toekomst in AZ Glorieux maar er moet rekening gehouden worden met requirements:
\begin{itemize}
    \item Eenvoudige requirements $\rightarrow$ Low-code voor citizen developers (no-code) - \textbf{PowerApps}.
    \item Complexe requirements $\rightarrow$ Low-code voor ontwikkelaars - \textbf{Outsystems}.
    \item Specifieke requirements, grote integratie met host systeem nodig $\rightarrow$ Traditionele desktop app.
\end{itemize}

Laat dit onderzoek daarom een leidraad zijn voor het maken van deze toekomstige keuzes.